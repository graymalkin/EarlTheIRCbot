\documentclass[12pt]{article}

\newcommand{\HRule}{\rule{\linewidth}{0.5mm}}
\begin{document}
\begin{titlepage}
\begin{center}

\textsc{\LARGE Earl The IRC Bot}\\[1.5cm]

\textsc{\Large An Extensible Erlang IRC Bot}\\[0.5cm]

\HRule \\[0.4cm]
{ \huge \bfseries Documentation \\[0.4cm] }
\HRule \\[1.5cm]

\noindent
\begin{minipage}{0.4\textwidth}
\begin{flushleft} \large
\emph{Earl Authors:}\\
Jonathan \textsc{Poole}\\
Simon \textsc{Cooksey}\\
\end{flushleft}

\end{minipage}%
\begin{minipage}{0.4\textwidth}
\begin{flushright} \large
\break
Martin \textsc{Ellis}\\
\end{flushright}
\end{minipage}

\vfill

% Bottom of the page
{\large \today}

\end{center}
\end{titlepage}

\section*{Introduction}

\subsection*{Abstract}

Earl is an open source, multi-process, extensible IRC bot writen in Erlang by
Jonathan Poole, Martin Ellis and Simon Cooksey.  His design has evolved as we've
learned about Erlang and OTP, and will likely continue to.

\subsection*{Build System}

Earl uses the standard erlang build system, and relx. The erlang.mk file will
download relx, and relx is used to generate the binary.


\subsection*{Basic Structure}

Earl has most of his functionality partitioned off into submodules. There are a
couple of essential ones, and some that are core to the IRC protocol.

The TCP/IP connection is handled with \texttt{earlConnection} module. This sends
data to a \texttt{buffer} (currenly found in the \texttt{earl} module), which
splits raw data into lines and passed onto the \texttt{messageRouter} module.

The \texttt{messageRouter} parses lines (using the \texttt{lineParse} module)
and creates creates gen events.

\subsection*{IRC Parser}

The IRC parser deals with most of the IRC protocol, and does the heavy lifting
as far as string manipulation goes.

The

\end{document}
