\documentclass[12pt]{article}

\begin{document}

\section*{Earl Documentation}

\subsection*{Basic Structure}

Earl has most of his functionality partitioned off into submodules. There are a
couple of essential ones, and some that are core to the IRC protocol.

The TCP/IP connection is handled with \texttt{earlConnection} module. This sends
data to a \texttt{buffer} (currenly found in the \texttt{earl} module), which
splits raw data into lines and passed onto the \texttt{messageRouter} module.

The \texttt{messageRouter} parses lines (using the \texttt{lineParse} module)
and creates creates gen events.

\subsection*{IRC Parser}

The IRC parser deals with most of the IRC protocol, and does the heavy lifting
as far as string manipulation goes.

\end{document}
