\documentclass[12pt]{article}

\newcommand{\HRule}{\rule{\linewidth}{0.5mm}}
\usepackage{listings}
\lstset{language=erlang}
\begin{document}
\begin{titlepage}
\begin{center}

\textsc{\LARGE Earl The IRC Bot}\\[1.5cm]

\textsc{\Large An Extensible Erlang IRC Bot}\\[0.5cm]

\HRule \\[0.4cm]
{ \huge \bfseries Documentation \\[0.4cm] }
\HRule \\[1.5cm]

\noindent
\begin{minipage}{0.4\textwidth}
\begin{flushleft} \large
\emph{Earl Authors:}\\
Jonathan \textsc{Poole}\\
Simon \textsc{Cooksey}\\
\end{flushleft}

\end{minipage}%
\begin{minipage}{0.4\textwidth}
\begin{flushright} \large
\break
Martin \textsc{Ellis}\\
\end{flushright}
\end{minipage}

\vfill

% Bottom of the page
{\large \today}

\end{center}
\end{titlepage}

\section*{Introduction}

\subsection*{Abstract}

Earl is an open source, multi-process, extensible IRC bot writen in Erlang by
Jonathan Poole, Martin Ellis and Simon Cooksey.  His design has evolved as we've
learned about Erlang and OTP, and will likely continue to.

\subsection*{Build System}

Earl uses the standard erlang build system, and relx. The erlang.mk file will
download relx, and relx is used to generate the binary.


\subsection*{Basic Structure}

Earl has most of his functionality partitioned off into submodules. There are a
couple of essential ones, and some that are core to the IRC protocol.

The TCP/IP connection is handled with \texttt{earlConnection} module. This sends
data to a \texttt{buffer} (currenly found in the \texttt{earl} module), which
splits raw data into lines and passed onto the \texttt{messageRouter} module.

The \texttt{messageRouter} parses lines (using the \texttt{lineParse} module)
and creates creates gen events.

\subsection*{IRC Parser}

The IRC parser deals with most of the IRC protocol, and does the heavy lifting
as far as string manipulation goes.

The IRC parser takes lines and evaluates to records for each type of event 
defined in the IRC protocol. These records have the various bits of data 
available in them. They are defined in \texttt{ircParser.hrl}, which should be
included in any modules which use them. This will in all likely hood be all 
modules.

\subsection*{Message Router}

The message router takes lines from the buffer, runs them through the parser 
and then creates \texttt{gen\_event} s for all the modules registered under it.

This part of the application also handles PING responses, as it is core to the
protocol to do so, and it fits nicely. This is counter to the design of the 
application though - so this may change (soon \texttrademark).

\begin{lstlisting}
parse(SendPid, PluginsNames) ->
    receive
	die ->
	    io:format("parserPid :: EXIT~n"),
	    lists:foreach(fun(Name) -> gen_event:delete_handler(irc_messages, 
	    	Name, []) end, PluginsNames),
	    exit(self(), normal);

    % deal with registerPlugin requests by adding them to the chan list
	#registerPlugin{name=Name} ->
	    ?MODULE:parse(SendPid, [load(Name)|PluginsNames]);

	% deregister plugins
	#deregisterPlugin{name=Name} ->
	    ?MODULE:parse(SendPid, unload(Name, PluginsNames));

	T->
	    Line = lineParse(T),
	    gen_event:notify(irc_messages, Line),
	    % Code for pings and "#modules" command cut out
	    end,
    ?MODULE:parse(SendPid, PluginsNames).
\end{lstlisting}

Plugins are in a list which is pushed around tail recursively, and is updated
with a pair of helped functions which are called from the
\texttt{#registerPlugin} and \texttt{#deregisterPlugin} patterns.

These messages are generated by the Earl admin module.

\end{document}
